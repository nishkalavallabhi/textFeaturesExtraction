\documentclass{article}

\addtolength{\headheight}{-3cm}
\addtolength{\textheight}{7cm}
\addtolength{\oddsidemargin}{-2cm}
\addtolength{\textwidth}{4cm}

\parindent0cm

\usepackage[utf8]{inputenc}
\usepackage{setspace}
\usepackage{url}

\pagestyle{empty}

\newcommand{\abox}{\raisebox{.6ex}[0ex][0ex]{\fbox{\phantom{\rule{.8ex}{.8ex}}}}}

\begin{document}

\begin{center}
TSLL TechXplorations 2017\\
\textbf{TextFeaturesExtractor: Python tool for text mining and corpus analysis}
\\  Contributors: Sowmya Vajjala and Sagnik Banerjee
\\ Iowa State University, USA
\\ (tool under development)
\end{center}

\paragraph{Background: } We are developing a small tool (command line interface, not graphical) to extract different kinds of textual features such as word, part-of-speech sequences and their frequencies and frequencies of various syntactic relations in a text, and we support different forms of pre-processing (stemming, spelling correction etc.). 

The goal of the tool is two-fold: 
\begin{enumerate}
\item support more linguistic corpus analysis 
\item support text mining researchers by providing a suite of text features they can use to benchmark text classification. 
The tool is still under development, and is currently seen as a single large python file with some documentation. Our goal is to release the code for public use by December.
\end{enumerate}

\paragraph{3 Takeaways from this session}
\begin{enumerate}
\item learning about how to extract different kinds of textual features beyond words and word sequences
\item walking through the process of how to extract such features
\item sharing the python code so that enthusiastic people can try to use it and give feedback for future improvement of the tool
\end{enumerate}

Current version of the code is available for download at: \url{https://goo.gl/gRF4X9}

\paragraph{Currently supported features (for any text file)}
\begin{itemize}
\item Extraction of word, character and POS n-grams for any n- and their frequencies
\item Skip grams (i.e., n-grams with gaps)
\item Word-POS mixed n-gram representations
\item Perform phrase level chunking and syntactic (dependency) parsing and collect frequencies of most common syntactic structures
\item Pre-processing: lowercasing, stemming, spelling correction, stop word removal
\end{itemize}

\paragraph{Planned Extensions: }
\begin{itemize}
\item Working on folders containing many files
\item Storing output in human readable formats
\item Storing output to be used as input for text mining and machine learning algorithms
\item Add support for vectorized representations of words, beyond n-grams
\end{itemize}

\paragraph{contact: } sowmya@iastate.edu
\end{document}